\documentclass[10pt, a4paper]{article}
% Marges réduites au maximum
\usepackage[left=0.4in,top=0.4in,right=0.4in,bottom=0.4in]{geometry}
\usepackage[parfill]{parskip}
\usepackage[hidelinks]{hyperref}
\usepackage{array}
\usepackage{contour}
\usepackage{enumitem}
\usepackage{fontspec}
\usepackage{linegoal}
\usepackage{setspace}
\usepackage{ulem}
\usepackage[dvipsnames]{xcolor}
\usepackage{graphicx}
\usepackage[absolute,overlay]{textpos}

\TPGrid{16}{24}

\pagestyle{empty}

\renewcommand{\ULdepth}{1.8pt}

% Nouvelle largeur de colonne pour les dates empilées
\newcommand{\firstcolwidth}{65pt}
\newcommand{\capitalletterspacing}{10}
\newcommand{\spaceaftersection}{0.8em}
\newcommand{\spaceaftername}{0.5em}
\newcommand{\myuline}[1]{%
  \uline{\phantom{#1}}%
  \llap{\contour{white}{#1}}%
}
\newcommand{\drawline}{\leavevmode\leaders\hrule height 1pt\hfill\kern 0pt}
\newcommand{\link}[2]{\myuline{\textcolor{black}{\href{#1}{#2}}}}

\newcommand{\newsection}[1]{
    \drawline
    \break
    \begin{minipage}[t]{\firstcolwidth}
        \addfontfeature{LetterSpace=\capitalletterspacing}
        \footnotesize{\MakeUppercase{#1}}
        \addfontfeature{LetterSpace=0}
    \end{minipage}
}
\newcommand{\newsectionspace}{
    \begin{minipage}[t]{\linegoal}
        \hfill
        \vspace{\spaceaftersection}
    \end{minipage}
}

% COMMANDE MODIFIÉE : Dates l'une en dessous de l'autre
\newcommand{\experience}[6]{
    \begin{minipage}[t]{\firstcolwidth}
        \footnotesize{#1 \\ \textemdash \\ #2}
    \end{minipage}
    \begin{minipage}[t]{\linegoal}
        \textbf{\normalsize{#3 \textemdash\ #4}}
        \hfill
        \footnotesize{#5}
        \begin{singlespace}
            \begin{flushleft}
                \footnotesize{#6}
            \end{flushleft}
        \end{singlespace}
    \end{minipage}
}
\newcommand{\project}[3]{
    \begin{minipage}[t]{\firstcolwidth}
        \footnotesize{#1}
    \end{minipage}
    \begin{minipage}[t]{\linegoal}
        \textbf{\normalsize{#2}}
        \begin{singlespace}
            \begin{flushleft}
                \footnotesize{#3}
            \end{flushleft}
        \end{singlespace}
    \end{minipage}
}
% COMMANDE MODIFIÉE : Dates l'une en dessous de l'autre pour l'éducation
\newcommand{\education}[4]{
    \begin{minipage}[t]{\firstcolwidth}
        \footnotesize{#1 \\ \textemdash \\ #2}
    \end{minipage}
    \begin{minipage}[t]{\linegoal}
        \textbf{\normalsize{#3}}
        \hfill
        \footnotesize{#4}
    \end{minipage}
}

\begin{document}
    \noindent
    \begin{minipage}[b]{0.3\textwidth}
        \includegraphics[height=0.5\textwidth]{cv_linkedin_gray.png}
    \end{minipage}
    \begin{minipage}[b]{0.6\textwidth}
        \Large{\textbf{Marc Giannuzzi, Data Engineer}} \\
        \vspace{\spaceaftername}
        \footnotesize{(+33) 6 95 64 64 54, marc.giannuzzi@gmail.com}
    \end{minipage}

    \newsection{Liens}
    \begingroup
        \begin{minipage}[t]{\linegoal}
            \footnotesize{\link{https://github.com/marcgiannuzzi}{GitHub}} \\
            \footnotesize{\link{https://www.linkedin.com/in/marc-giannuzzi-b098b216b/}{Linkedin}}
        \end{minipage}
    \endgroup

    \newsection{Experience}

    \newsectionspace
    \experience{Sept. 2025}{Aujourd'hui}{Waiyer}{Auto entrepreneur}{Paris, FR}{
        \textit{Contexte :}
        \link{https://www.waiyer.com/}{Waiyer} est une entreprise du digital spécialisée en data et en architecture cloud.
        Waiyer permet d'optimiser la gestion des données grâce à des solutions sur mesure, en utilisant les dernières technologies. 
        Son but est de concevoir et de mettre en œuvre des solutions efficaces et pérennes pour aider les entreprises à tirer le meilleur parti de leurs données.
    
        \begin{itemize}[nosep, leftmargin=*]
            \item Mise en place de l'architecture des données
            \item Orchestration de pipelines de données
            \item Création de dashboards analytiques
            \item Business Intelligence
            \item Software engineering : développement de scripts, APIs, outils d'automatisation
            \item Vision fonctionnelle : Définition des besoins, des KPIs, des dashboards adaptés
            \item Veille technologique : suivi des évolutions dans le domaine de la data, identification des opportunités d'innovation pour les clients
        \end{itemize}
    
        \textit{Compétences :} Kubernetes · Github Actions · Python · AWS S3 · SQL · MongoDB · PostgreSQL · PowerBI · Flutter · Go · Typescript · gRPC · Redis
    }

    \newsectionspace
    \experience{Avr. 2025}{Sept. 2025}{SFEIR}{Data Architect pour LCL}{Villejuif, FR}{
        \textit{Contexte :}
        \link{https://www.lcl.fr/}{LCL} (Le Crédit Lyonnais) est une banque française majeure, filiale du groupe Crédit Agricole, qui propose des services bancaires aux particuliers, professionnels et entreprises.
    
        Optimisation de l'organisation interne de l'entité SNI et mise en place des tableaux de bord analytiques pour son CODIR. 
        L'enjeu était de consolider les données issues de différentes sources, orchestrer et automatiser leur traitement via une architecture scalable et maintenable.
    
        \begin{itemize}[nosep, leftmargin=*]
            \item Audit du SI existant : analyse des stacks en place (bases, scripts, outils), comparaison PostgreSQL, MongoDB, Teradata
            \item Architecture technique hybride (on-premise first, comparaison et propositions d'équivalents Cloud GCP)
            \item Orchestration Kubernetes de scripts Python dockerisés. 
            \item Stockage via PostgreSQL (mode analytique) et S3
            \item Design des schémas de données (tables, colonnes, index). 
            \item Sélection des librairies Python (pandas, SQLAlchemy)
            \item Intégration des API Jira et autres progiciels dans les pipelines
            \item Dataviz : dashboards Power BI dynamiques pour le CODIR. 
            \item Encadrement technique : collaboration avec 2 data engineers et 1 PO, revue technique et support à l'implémentation
            \item Bonnes pratiques, guides centralisés dans Confluence (architecture, code)
        \end{itemize}
    
        \textit{Compétences :} Kubernetes · GCP · GitLab CI/CD · Python · AWS S3 · SQL · MongoDB · PostgreSQL · Teradata · PowerBI · Elastic Stack (ELK)
    }

    \newsectionspace
    \experience{Mar. 2025}{Sept. 2025}{SFEIR}{Data Consultant}{Neuilly Sur Seine, FR}{
        \textit{Contexte :}
        \link{https://sfeir.com}{SFEIR} est une entreprise française, premier partenaire Google Cloud Platform GCP (Services 2023, Training 2024, EMEA 2025), spécialisée dans le développement de logiciels et l'accompagnement technologique des entreprises.
        
        En charge de \textbf{recrutements} de profils techniques: Data Engineers, Data Scientists, ML Ops, Devops, Fullstack
        
        \textit{Compétences :} GCP (Google Cloud Platform)
    }

    \newsectionspace
    \experience{Jan. 2023}{Jan. 2025}{Visian}{Data Consultant pour Engie}{Paris, FR}{
        \textit{Contexte :}
        \link{https://particuliers.engie.fr/}{Engie} est le premier fournisseur de gaz naturel et d'électricité verte en France.

        \textbf{Data Engineer} faisant partie de l'équipe agile (5 personnes) développant les produit Mon Bonus Engie / Ecodefi+, ICE (Impact Carbone Elec) et Ecogestes Dynamiques évoluant dans le framework SAFe
        \begin{itemize}[nosep, leftmargin=*]
            \item Ecodefi+ \link{https://particuliers.engie.fr/electricite/contrat-electricite/mon-bonus-engie.html}{(Mon Bonus Engie)} est un service gratuit proposant aux particuliers d'être récompensé sur leur sobriété énergétique lors de challenges (\textbf{300k clients})
            \item ICE \link{https://particuliers.engie.fr/decouvrir-espace-client/application-mobile/meteo-carbone-elec.html}{(Impact Carbone Elec)} permet aux clients d'\textbf{analyser les pics de consommation carbone dans le futur et d'adapter leur consommation} en fonction (\textbf{1M clients}).
            \item Ecogestes Dynamiques contribue à expliquer les impacts des objets électriques dans les foyers (\textbf{1M clients})
        \end{itemize}

        \textbf{Missions}
        \begin{itemize}[nosep, leftmargin=*]
            \item Implémentation et développement des produits
            \item Gestion des données à grande échelle (bases des données, orchestration des flux...)
            \item Elaboration des KPIs techniques et fonctionnelles
            \item Monitoring et Alerting des solutions
            \item Création d'APIs REST (100k appels/jour), latence < 1sec
            \item Participation à l'architecture des micro-services AWS et des plateformes digitales
            \item Optimisations (mémoire, temps d'exécution, refactoring...)
            \item Optimisations des flux bases de données (structure, index...)
            \item Reporting automatisé via SNS, Gitlab CICD
            \item Rédaction de roadmaps dans le cadre de la migration d'une technologie centrale au sein du train SAFe
            \item Participations aux architectures technico-fonctionnelles
            \item Scaling d'algorithmes de Deep Learning (DeepAR, Temporal Fusion Transformer...) 
        \end{itemize}

        \textit{Compétences :} c3ai · JavaScript · Jenkins · AWS Lambda · Amazon Kinesis · Databricks · Elastic Stack (ELK) · Python · AWS S3 · Terraform · Amazon CloudWatch · Splunk · MLflow · Amazon DynamoDB · TIDB · Apache Airflow · SQS · SNS
    }
        
    \newsectionspace
    \experience{Jan. 2023}{Jan. 2025}{Visian}{Data Consultant}{La Défense, FR}{
        \textit{Contexte :}
        \link{https://visian.tech/}{Visian} est une entreprise de \textbf{conseil} spécialisée dans les projets utilisant la donnée à grande échelle.
        Ma présentation vidéo de l'entreprise et du métier: \link{https://www.youtube.com/watch?v=X60fo2I4ltw&ab_channel=WelcometotheJungle}{Youtube}
        \begin{itemize}[nosep, leftmargin=*]
            \item Recrutements de profils techniques (Data Ingénieurs, Data Scientists, Devops, Fullstack)
            \item \textbf{Direction du projet Pysquad}, développant en interne des projets digitaux
            \item Coordination de l'implémentation générique d'un \textbf{projet d'IA générative} pour différentes tâches d'inférence avec Azure AI Studio
            \item \textbf{Création d'une data plateforme} en vue d'aider les sales à récupérer les données de missions, prospects, clients, consultants etc
        \end{itemize}
        \textit{Compétences :} Docker · Azure AI Studio · Django · Streamlit · PowerBI · FastAPI · Elastic Stack (ELK) · AWS RDS · Redis · AWS Code Artifacts
    }

    \newsectionspace
    \experience{Août 2021}{Oct. 2022}{Aqua Digital Rising}{Lead Data Scientist}{Paris, FR}{
        \textit{Contexte :}
        \link{https://www.linkedin.com/company/aqua-digital-rising/}{Aqua Digital Rising} est une \textbf{startup fintech} qui ambitionne de révolutionner le pari sportif.

        \textbf{Gestion de la donnée}
        \begin{itemize}[nosep, leftmargin=*]
            \item Gestion du stockage (MongoDB, SQL server, HDF5) à partir de Sport Radar et Opta 
            \item Structuration des flux de données en Python et optimisation de la vitesse de lecture grâce aux index SQL Server
            \item Création d'une API REST hautement configurable pour l'équipe front-end 
            \item Sauvegardes temporaires via Redis et mise à jour en direct avec RabbitMq 
            \item Analyses, data visualisation et algorithmes de machine learning (clustering)
        \end{itemize}
    
        \textbf{Autres missions}
        \begin{itemize}[nosep, leftmargin=*]
            \item Configuration divers services Azure (Functions, Web Services, Artifacts, Pipelines, Keyvault) 
            \item Analyses quantitatives de crypto-monnaies avec InfluxDb pour alimenter la stratégie d'investissement 
            \item Coordination de 2 ingénieurs (un data scientist et un développeur front-end)
        \end{itemize}

        \textit{Compétences :} Microsoft SQL Server · Scikit-learn · Redis · JIRA · RabbitMQ · Bases de données · InfluxDB · Bash · MongoDB · Microsoft Azure
    }

    \newsectionspace
    \experience{Fév. 2021}{Juil. 2021}{Orange Business Services}{Data Scientist}{Paris, FR}{
        \textit{Contexte :} Société de \textbf{conseil} dans la transformation digitale dans des projets de communication.
        \begin{itemize}[nosep, leftmargin=*]
            \item Analyses et diagnostics des anomalies du réseau VoIP via machine learning (clustering)
            \item Création d'un graphe de réseau VoIP dynamique avec Kibana puis Neo4J intégré dans une page web
            \item Analyses des métriques Devops d'un cluster de serveurs Elastic
            \item Automatisation de la suppression des logs Elastic avec log4j
        \end{itemize}
        \textit{Compétences :} HTML · Scikit-learn · Elastic Stack (ELK) · CSS · Python · Neo4j · JavaScript · Bases de données · Bash
    }

    \newsectionspace
    \experience{Août 2020}{Mars 2021}{Digiberries}{Software Engineer}{Paris, FR}{
        \textit{Contexte :} Agence de référencement web, améliorant les stratégies SEO des clients.
        \begin{itemize}[nosep, leftmargin=*]
            \item Développement d'un script d'automatisation de création de rapports excel (Google Ads)
            \item Création d'un outil d'analyse SEO et rapport automatisé PDF avec Lighthouse et NodeJS
            \item Automatisation d'envoi de mails
        \end{itemize}
        \textit{Compétences :} Lighthouse · Python · Google Ads · Node.js
    }

    \newsectionspace
    \experience{Avr. 2020}{Août 2020}{Eudonet CRM}{Data Scientist}{Courbevoie, FR}{
        \textit{Contexte :} Développe des logiciels CRM dans un context B2B.
        \begin{itemize}[nosep, leftmargin=*]
            \item Mise en place de la suite Elastic (ELK + Filebeat) pour l'analyse de logs Windows IIS
            \item Création d'un outil de monitoring et algorithme Python de détection de baisse d'utilisation (FBProphet)
            \item Création d'une plateforme web (Javascript, Node.js) de visualisation dynamique
            \item Visualisations PowerBI interne et création de scripts powershell pour la gestion d'applications IIS
            \item Requêtes SQL pour rapports et modification de paramètres, sprints agiles hebdomadaires
        \end{itemize}
        \textit{Compétences :} Microsoft SQL Server · HTML · PowerShell · ELK · Python · Power BI · JavaScript · Bash
    }

    \newsectionspace
    \experience{Juil. 2019}{Sep. 2019}{Veesion}{Software Engineer}{Paris}{
        \textit{Contexte :} Scale-up française de détection automatique de vols via IA.
        \begin{itemize}[nosep, leftmargin=*]
            \item Création d'une plateforme interne de visualisation et labellisation vidéo avec Django
            \item Classification de données pour l'apprentissage du modèle de computer vision
            \end{itemize}
        \textit{Compétences :} HTML · CSS · Python · JavaScript · Amazon S3 · Django
    }

    \newsection{Projets personnels}
    
    \vspace{0.4em}
    \noindent
    \begin{minipage}[t]{\firstcolwidth}
        \hfill
    \end{minipage}
    \begin{minipage}[t]{\linegoal}
        \begin{itemize}[nosep, leftmargin=*]
            \item \textbf{Intelligence Artificielle} : Création automatisée de contenus digitaux multimodaux
            \item \textbf{Finance} : Algorithmes et visualisations d'aide à l'analyse financière (Python)
            \item \textbf{Algorithmes de tri} : Plateforme de visualisation d'algorithmes de tri (React)
            \item \textbf{NLP} : Agent conversationnel lyrics de chanson (Chatbot Messenger)
            \item \textbf{Computer Vision} : Transformation des gestes en équations (Tensorflow, OpenCV)
            \item \textbf{Deep Learning} : Algorithme de scoring santé de recettes cuisine (Tensorflow)
        \end{itemize}
    \end{minipage}

    \newsection{Certifications}
    
    \newsectionspace
    \project{2025}{GCP - Associate Cloud Engineer / Professional Data Engineer}{}

    \newsection{Education}
    
    \newsectionspace
    \education{Sep. 2018}{Juin 2021}{ESILV (École Supérieure d'Ingénieurs Léonard De Vinci)}{Courbevoie, FR}
    \begin{itemize}[nosep, leftmargin=*]
        \item Master en Informatique, Big Data et Objets Connectés - Spécialisation en Data Science
    \end{itemize}

    \newsectionspace
    \education{Sep. 2019}{Jan. 2020}{UCR (University Of California Riverside)}{Riverside, CA, USA}
    \begin{itemize}[nosep, leftmargin=*]
        \item Data Mining Techniques (CS 235), Machine Learning (CS 171), Python programming (CS 009)
    \end{itemize}
    
    \newsectionspace
    \education{Sep. 2016}{Déc. 2018}{CPGE Fénelon Sainte-Marie}{Paris, FR}
    \begin{itemize}[nosep, leftmargin=*]
        \item MPSI / MP (Classe Préparatoire aux Grandes Écoles)
    \end{itemize}

    \newsection{Langages}
    \begin{minipage}[t]{\linegoal}
        \small{\textbf{English}}: Fluent \qquad
        \small{\textbf{Français}}: Natif \qquad
    \end{minipage}
\end{document}