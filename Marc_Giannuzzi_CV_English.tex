\documentclass[10pt, a4paper]{article}
\usepackage[left=0.6in,top=0.4in,right=0.6in,bottom=0.4in]{geometry}
\usepackage[parfill]{parskip}
\usepackage[hidelinks]{hyperref}
\usepackage{array}
\usepackage{contour}
\usepackage{enumitem}
\usepackage{fontspec}
\usepackage{linegoal}
\usepackage{setspace}
\usepackage{ulem}
\usepackage[dvipsnames]{xcolor}
\usepackage{graphicx}
\usepackage[absolute,overlay]{textpos}
\TPGrid{16}{24}  % Creates a 16-column, 24-row grid on the page
\pagestyle{empty}
\renewcommand{\ULdepth}{1.8pt}
\newcommand{\firstcolwidth}{120pt}
\newcommand{\capitalletterspacing}{10}
\newcommand{\spaceaftersection}{1em}
\newcommand{\spaceaftername}{0.5em}
\newcommand{\myuline}[1]{%
\uline{\phantom{#1}}%
\llap{\contour{white}{#1}}%
}
\newcommand{\drawline}{\leavevmode\leaders\hrule height 1pt\hfill\kern 0pt}
\newcommand{\link}[2]{\myuline{\textcolor{black}{\href{#1}{#2}}}}
\newcommand{\newsection}[1]{
\drawline
\break
\begin{minipage}[t]{\firstcolwidth}
\addfontfeature{LetterSpace=\capitalletterspacing}
\footnotesize{\MakeUppercase{#1}}
\addfontfeature{LetterSpace=0}
\end{minipage}
}
\newcommand{\newsectionspace}{
\begin{minipage}[t]{\linegoal}
\hfill
\vspace{\spaceaftersection}
\end{minipage}
}
\newcommand{\experience}[6]{
\begin{minipage}[t]{\firstcolwidth}
\footnotesize{#1 \textemdash\ #2}
\end{minipage}
\begin{minipage}[t]{\linegoal}
\textbf{\normalsize{#3 \textemdash\ #4}}
\hfill
\footnotesize{#5}
\begin{singlespace}
\begin{flushleft}
\footnotesize{#6}
\end{flushleft}
\end{singlespace}
\end{minipage}
}
\newcommand{\project}[3]{
\begin{minipage}[t]{\firstcolwidth}
\footnotesize{#1}
\end{minipage}
\begin{minipage}[t]{\linegoal}
\textbf{\normalsize{#2}}
\begin{singlespace}
\begin{flushleft}
\footnotesize{#3}
\end{flushleft}
\end{singlespace}
\end{minipage}
}
\newcommand{\award}[5]{
\begin{minipage}[t]{\firstcolwidth}
\footnotesize{#1}
\end{minipage}
\begin{minipage}[t]{\linegoal}
\textbf{\normalsize{#2, #3}}
\hfill
\footnotesize{#4}
\begin{singlespace}
\begin{flushleft}
\footnotesize{#5}
\end{flushleft}
\end{singlespace}
\end{minipage}
}
\newcommand{\education}[4]{
\begin{minipage}[t]{\firstcolwidth}
\footnotesize{#1 \textemdash\ #2}
\end{minipage}
\begin{minipage}[t]{\linegoal}
\textbf{\normalsize{#3}}
\hfill
\footnotesize{#4}
\end{minipage}
}
\begin{document}
\begin{minipage}[b]{0.3\textwidth}
\includegraphics[height=0.5\textwidth]{cv_linkedin_gray.png}
\end{minipage}
\begin{minipage}[b]{0.6\textwidth}
\Large{\textbf{Marc Giannuzzi, Software Engineer}} \
\vspace{\spaceaftername}
\footnotesize{(+33)695646454, marc.giannuzzi@gmail.com}
\end{minipage}


\newsection{Links}
\begingroup
    \begin{minipage}[t]{\linegoal}
        \footnotesize{\link{https://github.com/marcgiannuzzi}{GitHub}} \\
        \footnotesize{\link{https://www.linkedin.com/in/marc-giannuzzi-b098b216b/}{Linkedin}}
    \end{minipage}
\endgroup

\newsection{Experience}
\newsectionspace
\experience{January 2023}{\textit{present}}{Visian}{Machine Learning Engineer}{La Défense, FR}{
    \textit{Context:}
    \link{https://visian.tech/}{Visian} is a \textbf{consulting} company specializing in large-scale data projects.
    My video presentation: \link{https://www.youtube.com/watch?v=X60fo2I4ltw&ab_channel=WelcometotheJungle}{Youtube}
    \begin{itemize}
        \item Recruitment of technical profiles (Data Engineers, Data Scientists, Devops, Fullstack)
        \item \textbf{Leadership of the Pysquad project}, developing digital projects internally
        \item Coordination of the generic implementation of a \textbf{generative Artificial Intelligence} (AI) project for various inference tasks (textual, vision, multimodal etc...) using Azure ML/DL services and Azure AI Studio
        \item \textbf{Creation of a data platform} to help sales retrieve data on missions, prospects, clients, consultants etc.
        
    \end{itemize}
    \textit{Skills:} Docker · Azure AI Studio · Django · Streamlit · PowerBI · FastAPI · Elastic Stack (ELK) · AWS RDS · Redis · AWS Code Artifacts
    
    \newsectionspace
    \experience{January 2023}{\textit{present}}{Engie}{Data Consultant}{Paris, FR}{
        \textit{Context:}
        \link{https://particuliers.engie.fr/}{Engie} is the leading supplier of natural gas and green electricity, as well as the third largest global group in the energy sector.

        \textbf{Data Engineer} part of the agile team (5 people) developing the products Mon Bonus Engie / Ecodefi+, ICE (Impact Carbone Elec) and Dynamic Eco-gestures evolving in the SAFe framework
        \begin{itemize}
            \item Ecodefi+ \link{https://particuliers.engie.fr/electricite/contrat-electricite/mon-bonus-engie.html}{(Mon Bonus Engie)} is a free service offering individuals rewards for their energy sobriety during challenges (\textbf{300k clients})
            
            \item ICE \link{https://particuliers.engie.fr/decouvrir-espace-client/application-mobile/meteo-carbone-elec.html}{(Impact Carbone Elec)} allows clients to \textbf{analyze future carbon consumption peaks and adapt their consumption} accordingly (\textbf{1M clients}).
            \item Dynamic Eco-gestures helps explain the impacts of electrical objects in households (\textbf{1M clients})
        \end{itemize}

        \textbf{Missions}
        \begin{itemize}
            \item Implementation and development of products
            \item Large-scale data management (databases, flow orchestration...)
            \item Development of technical and functional KPIs
            \item Monitoring and Alerting of solutions
            \item Creation of REST APIs (100k calls/day), latency < 1sec
            \item Scaling of Deep Learning algorithms (DeepAR, Temporal Fusion Transformer...) 
            \item Participation in AWS microservices architecture and digital platforms
            \item Optimizations (memory, execution time, refactoring...)
            \item Database flow optimizations (structure, index...)
            \item Automated reporting via SNS
            \item Gitlab CICD
            \item Writing roadmaps as part of the migration of a central technology used for several years within the SAFe train
            \item Participation in technical-functional architectures
        \end{itemize}

        \textit{Skills:} c3ai · JavaScript · Jenkins · AWS Lambda · Amazon Kinesis · Databricks · Elastic Stack (ELK) · Python (programming language) · AWS S3 · Terraform · Amazon CloudWatch · Splunk · MLflow · Amazon DynamoDB · TIDB · Apache Airflow · Amazon Simple Queue Service (SQS) · Amazon Simple Notification Service (SNS)
    }
}
\newsectionspace
\experience{August 2021}{October 2022}{Aqua Digital Rising}{Lead Data Scientist}{Paris, FR}{
    \textit{Context:}
    \link{https://www.linkedin.com/company/aqua-digital-rising/}{Aqua Digital Rising} is a \textbf{fintech startup} that aims to revolutionize sports betting by analyzing all player and team statistics.

    \textbf{Data management}
    \begin{itemize}
        \item Management of storage (creation, structuring, cleaning) of several databases (MongoDB, SQL server, HDF5) from data retrieved by Sport Radar and Opta providers 
        \item Structuring of data flows in Python through the different components of the architecture and optimization of reading speed thanks to SQL Server indexes (several hundred GB)
        \item Creation of a highly configurable REST API allowing the front-end team to retrieve database aggregations and other static information 
        \item Temporary backups of API query results using Redis 
        \item Real-time updates of statistical data using RabbitMq 
        \item Analyses and data visualization
        \item Machine learning algorithms (clustering) and visualizations for analyzing player and team statistics
    \end{itemize}

    \textbf{Other missions}
    \begin{itemize}
        \item Configuration of various Azure services (Azure Functions, Azure Web Services, Azure Artifacts, Azure Pipelines, Azure Keyvault) in line with the company's Cloud strategy 
        \item Quantitative analyses of cryptocurrencies with InfluxDb to inform investment strategy 
        \item Coordination of 2 engineers (a data scientist and a front-end developer)
    \end{itemize}

    \textit{Skills:} Microsoft SQL Server · Scikit-learn · Redis · JIRA · RabbitMQ · Databases · InfluxDB · Bash · MongoDB · Microsoft Azure
}
\newsectionspace
\experience{February 2021}{July 2021}{Orange Business Services}{Data Scientist}{Paris, FR}{
    \textit{Context:}
    \link{https://www.orange-business.com/fr}{Orange Business Services} is a \textbf{consulting} company in digital transformation for communication projects.
    \begin{itemize}
        \item Analyses and diagnostics of VoIP network anomalies, particularly with machine learning (clustering)
        \item Creation of a dynamic VoIP network graph with Kibana then Neo4J for easier visualization, integrated into a web page
        \item Analyses of Devops metrics (CPU usage, storage...) of an Elastic server cluster
        \item Automation of Elastic log deletion with log4j
    \end{itemize}

    \textbf{Note:}
    VoIP = Voice over Internet Protocol is a network that allows phone calls via the Internet


    \textit{Skills:} HTML · Scikit-learn · Elastic Stack (ELK) · Cascading Style Sheets (CSS) · Python (programming language) · Neo4j · JavaScript · Databases · Bash
}
\newsectionspace
\experience{August 2020}{March 2021}{Digiberries}{Software Engineer}{Paris, FR}{
    \textit{Context:}
    \link{https://digiberries.fr/}{Digiberries} is a \textbf{web referencing agency}, improving clients' SEO strategies.
    \begin{itemize}
        \item Development of a script allowing automation of Excel report creation from Google Ads data
        \item Creation of an SEO analysis tool and automated PDF report creation with Lighthouse and NodeJS packages
        \item Email sending automation
    \end{itemize}
    \textit{Skills:} Lighthouse · Python (programming language) · Google Ads · Node.js
}
\newsectionspace
\experience{April 2020}{August 2020}{Eudonet CRM}{Data Scientist}{Courbevoie, FR}{
    \textit{Context:}
    \link{https://fr.eudonet.com/}{Eudonet} \textbf{develops CRM software} in a B2B context.
    \begin{itemize}
        \item Implementation of the Elastic suite (ElasticSearch, Kibana, Logstash + Filebeat) for analyzing Windows IIS log files
        \item Creation of a monitoring tool with dynamic dashboards in Kibana
        \item Creation of a Python algorithm for detecting decrease in CRM usage by clients. Use of the FBProphet library
        \item Creation of a dynamic web platform (Javascript, Node.js) allowing visualization of the results of the previous algorithm
        \item PowerBI visualizations for internal management of employee projects
        \item Creation of powershell scripts for managing IIS applications
        \item SQL queries for creating reports on client usage and modifying parameters
        \item Weekly agile sprints
    \end{itemize}
    \textit{Skills:} Microsoft SQL Server · HTML · Windows PowerShell · Elastic Stack (ELK) · Cascading Style Sheets (CSS) · Python (programming language) · Microsoft Power BI · JavaScript · Databases · Bash
}
\newsectionspace
\experience{July 2019}{Sept. 2019}{Veesion}{Software Engineer}{Paris}{
    \link{https://veesion.io/}{Veesion} is a French scale-up aiming to \textbf{automatically detect thefts} in supermarkets.
    \begin{itemize}
        \item Creation of an internal platform allowing real-time visualization and labeling 
            (to improve the company's theft detection algorithm) of videos
            with Django
        \item Classification of data allowing the learning of the computer vision model
            Veesion is a French scale-up aiming to automatically detect thefts in supermarkets. \item Creation of an internal platform allowing real-time visualization and labeling (to improve the company's theft detection algorithm) of videos with Django \item Classification of data allowing the learning of the computer vision model
        \end{itemize}
    
    \textit{Skills:} HTML · Cascading Style Sheets (CSS) · Python (programming language) · JavaScript · Amazon S3 · Django
}

    \newsection{Projects}
    
    \newsectionspace
    \project{2024}{Artificial Intelligence}{
        Automated creation of multimodal digital content
    }
    \newsectionspace
    \project{2024}{Finance}{
        Algorithms and visualizations to aid financial analysis (Python)
    }
    \newsectionspace
    \project{2021}{Sorting Algorithms}{
        Platform for visualizing sorting algorithms (React)
    }
    \newsectionspace
    \project{2021}{Web Semantics}{
        Website displaying the availability of bikes/trains (RDF)
    }
    \newsectionspace
    \project{2020}{NLP (Natural Language Processing)}{
        Conversational agent displaying song lyrics (Chatbot Messenger)
    }
    \newsectionspace
    \project{2020}{Computer vision}{
        Computer vision algorithm transforming finger gestures into equations via a webcam (Tensorflow, OpenCV)
    }
    \newsectionspace
    \project{2019}{Deep Learning}{
        Algorithm for health scoring of cooking recipes (Tensorflow)
    }

    \newsection{Education}
    
    \newsectionspace
    \education{Sept. 2018}{June 2021}{ESILV (École Supérieure D'ingénieurs Léonard De Vinci)}{Courbevoie, FR}
    \begin{itemize}
        \item Master's in Computer Science, Big Data, and IoT
        \item Specialization in Data Science
    \end{itemize}

    \newsectionspace
    \education{Sept. 2019}{Jan 2020}{UCR (University Of California Riverside)}{Riverside, California USA}
    \begin{itemize}
        \item {\link{https://msol.ucr.edu/courses/cs235}{CS 235}} Data Mining Techniques
        \item {\link{https://www.cs.ucr.edu/~epapalex/teaching/171_W22/index.html}{CS 171}} Introduction to Machine Learning and Data Mining
        \item {CS 009} Python programming
    \end{itemize}
    
    \newsectionspace
    \education{Sept. 2016}{Dec. 2018}{CPGE Fénelon Sainte-Marie}{Paris, FR}
    \begin{itemize}
        \item MPSI
        \item MP
    \end{itemize}


    \newsection{Languages}
    \begin{minipage}[t]{\linegoal}
        \small{\textbf{English}}: Fluent \qquad
        \small{\textbf{Français}}: Native \qquad
    \end{minipage}
\end{document}
